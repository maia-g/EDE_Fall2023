% Options for packages loaded elsewhere
\PassOptionsToPackage{unicode}{hyperref}
\PassOptionsToPackage{hyphens}{url}
%
\documentclass[
]{article}
\usepackage{amsmath,amssymb}
\usepackage{iftex}
\ifPDFTeX
  \usepackage[T1]{fontenc}
  \usepackage[utf8]{inputenc}
  \usepackage{textcomp} % provide euro and other symbols
\else % if luatex or xetex
  \usepackage{unicode-math} % this also loads fontspec
  \defaultfontfeatures{Scale=MatchLowercase}
  \defaultfontfeatures[\rmfamily]{Ligatures=TeX,Scale=1}
\fi
\usepackage{lmodern}
\ifPDFTeX\else
  % xetex/luatex font selection
\fi
% Use upquote if available, for straight quotes in verbatim environments
\IfFileExists{upquote.sty}{\usepackage{upquote}}{}
\IfFileExists{microtype.sty}{% use microtype if available
  \usepackage[]{microtype}
  \UseMicrotypeSet[protrusion]{basicmath} % disable protrusion for tt fonts
}{}
\makeatletter
\@ifundefined{KOMAClassName}{% if non-KOMA class
  \IfFileExists{parskip.sty}{%
    \usepackage{parskip}
  }{% else
    \setlength{\parindent}{0pt}
    \setlength{\parskip}{6pt plus 2pt minus 1pt}}
}{% if KOMA class
  \KOMAoptions{parskip=half}}
\makeatother
\usepackage{xcolor}
\usepackage[margin=1in]{geometry}
\usepackage{color}
\usepackage{fancyvrb}
\newcommand{\VerbBar}{|}
\newcommand{\VERB}{\Verb[commandchars=\\\{\}]}
\DefineVerbatimEnvironment{Highlighting}{Verbatim}{commandchars=\\\{\}}
% Add ',fontsize=\small' for more characters per line
\usepackage{framed}
\definecolor{shadecolor}{RGB}{248,248,248}
\newenvironment{Shaded}{\begin{snugshade}}{\end{snugshade}}
\newcommand{\AlertTok}[1]{\textcolor[rgb]{0.94,0.16,0.16}{#1}}
\newcommand{\AnnotationTok}[1]{\textcolor[rgb]{0.56,0.35,0.01}{\textbf{\textit{#1}}}}
\newcommand{\AttributeTok}[1]{\textcolor[rgb]{0.13,0.29,0.53}{#1}}
\newcommand{\BaseNTok}[1]{\textcolor[rgb]{0.00,0.00,0.81}{#1}}
\newcommand{\BuiltInTok}[1]{#1}
\newcommand{\CharTok}[1]{\textcolor[rgb]{0.31,0.60,0.02}{#1}}
\newcommand{\CommentTok}[1]{\textcolor[rgb]{0.56,0.35,0.01}{\textit{#1}}}
\newcommand{\CommentVarTok}[1]{\textcolor[rgb]{0.56,0.35,0.01}{\textbf{\textit{#1}}}}
\newcommand{\ConstantTok}[1]{\textcolor[rgb]{0.56,0.35,0.01}{#1}}
\newcommand{\ControlFlowTok}[1]{\textcolor[rgb]{0.13,0.29,0.53}{\textbf{#1}}}
\newcommand{\DataTypeTok}[1]{\textcolor[rgb]{0.13,0.29,0.53}{#1}}
\newcommand{\DecValTok}[1]{\textcolor[rgb]{0.00,0.00,0.81}{#1}}
\newcommand{\DocumentationTok}[1]{\textcolor[rgb]{0.56,0.35,0.01}{\textbf{\textit{#1}}}}
\newcommand{\ErrorTok}[1]{\textcolor[rgb]{0.64,0.00,0.00}{\textbf{#1}}}
\newcommand{\ExtensionTok}[1]{#1}
\newcommand{\FloatTok}[1]{\textcolor[rgb]{0.00,0.00,0.81}{#1}}
\newcommand{\FunctionTok}[1]{\textcolor[rgb]{0.13,0.29,0.53}{\textbf{#1}}}
\newcommand{\ImportTok}[1]{#1}
\newcommand{\InformationTok}[1]{\textcolor[rgb]{0.56,0.35,0.01}{\textbf{\textit{#1}}}}
\newcommand{\KeywordTok}[1]{\textcolor[rgb]{0.13,0.29,0.53}{\textbf{#1}}}
\newcommand{\NormalTok}[1]{#1}
\newcommand{\OperatorTok}[1]{\textcolor[rgb]{0.81,0.36,0.00}{\textbf{#1}}}
\newcommand{\OtherTok}[1]{\textcolor[rgb]{0.56,0.35,0.01}{#1}}
\newcommand{\PreprocessorTok}[1]{\textcolor[rgb]{0.56,0.35,0.01}{\textit{#1}}}
\newcommand{\RegionMarkerTok}[1]{#1}
\newcommand{\SpecialCharTok}[1]{\textcolor[rgb]{0.81,0.36,0.00}{\textbf{#1}}}
\newcommand{\SpecialStringTok}[1]{\textcolor[rgb]{0.31,0.60,0.02}{#1}}
\newcommand{\StringTok}[1]{\textcolor[rgb]{0.31,0.60,0.02}{#1}}
\newcommand{\VariableTok}[1]{\textcolor[rgb]{0.00,0.00,0.00}{#1}}
\newcommand{\VerbatimStringTok}[1]{\textcolor[rgb]{0.31,0.60,0.02}{#1}}
\newcommand{\WarningTok}[1]{\textcolor[rgb]{0.56,0.35,0.01}{\textbf{\textit{#1}}}}
\usepackage{longtable,booktabs,array}
\usepackage{calc} % for calculating minipage widths
% Correct order of tables after \paragraph or \subparagraph
\usepackage{etoolbox}
\makeatletter
\patchcmd\longtable{\par}{\if@noskipsec\mbox{}\fi\par}{}{}
\makeatother
% Allow footnotes in longtable head/foot
\IfFileExists{footnotehyper.sty}{\usepackage{footnotehyper}}{\usepackage{footnote}}
\makesavenoteenv{longtable}
\usepackage{graphicx}
\makeatletter
\def\maxwidth{\ifdim\Gin@nat@width>\linewidth\linewidth\else\Gin@nat@width\fi}
\def\maxheight{\ifdim\Gin@nat@height>\textheight\textheight\else\Gin@nat@height\fi}
\makeatother
% Scale images if necessary, so that they will not overflow the page
% margins by default, and it is still possible to overwrite the defaults
% using explicit options in \includegraphics[width, height, ...]{}
\setkeys{Gin}{width=\maxwidth,height=\maxheight,keepaspectratio}
% Set default figure placement to htbp
\makeatletter
\def\fps@figure{htbp}
\makeatother
\setlength{\emergencystretch}{3em} % prevent overfull lines
\providecommand{\tightlist}{%
  \setlength{\itemsep}{0pt}\setlength{\parskip}{0pt}}
\setcounter{secnumdepth}{-\maxdimen} % remove section numbering
\ifLuaTeX
  \usepackage{selnolig}  % disable illegal ligatures
\fi
\IfFileExists{bookmark.sty}{\usepackage{bookmark}}{\usepackage{hyperref}}
\IfFileExists{xurl.sty}{\usepackage{xurl}}{} % add URL line breaks if available
\urlstyle{same}
\hypersetup{
  pdftitle={A06 - Crafting Reports},
  pdfauthor={Maia Griffith},
  hidelinks,
  pdfcreator={LaTeX via pandoc}}

\title{A06 - Crafting Reports}
\author{Maia Griffith}
\date{Fall 2023}

\begin{document}
\maketitle

{
\setcounter{tocdepth}{3}
\tableofcontents
}
\hypertarget{objectives}{%
\subsection{Objectives:}\label{objectives}}

\begin{enumerate}
\def\labelenumi{\arabic{enumi}.}
\tightlist
\item
  More practice with R code chunk options
\item
  Gain proficiency with figures, tables (w/\texttt{Kable}) table of
  contents, etc.
\item
  Debugging knitting issues
\end{enumerate}

\hypertarget{directions}{%
\subsection{Directions}\label{directions}}

\begin{enumerate}
\def\labelenumi{\arabic{enumi}.}
\tightlist
\item
  Rename this file
  \texttt{\textless{}FirstLast\textgreater{}\_A06\_CraftingReports.Rmd}
  (replacing \texttt{\textless{}FirstLast\textgreater{}} with your first
  and last name).
\item
  Change ``Student Name'' on line 3 (above) with your name.
\item
  Work through the tasks, \textbf{creating code and output} that fulfill
  each instruction.
\item
  Be sure your code is tidy; use line breaks to ensure your code fits in
  the knitted output.
\item
  Be sure to \textbf{answer the questions} in this assignment document.
\item
  When you have completed the assignment, \textbf{Knit} the text and
  code into a single PDF file.
\item
  \textbf{Be sure that you also commit and push your final Rmd document
  to your GitHub account}.
\end{enumerate}

\hypertarget{task-1---basic-markdown}{%
\subsection{Task 1 - Basic Markdown}\label{task-1---basic-markdown}}

Using markdown, create a table beneath the
\texttt{Table:\ EPA\ Air\ Quality} line below that summarizes the
metadata of the EPA Air Quality data. The first column should have the
header ``Item'' and should include the the three metadata attribute item
names: ``Source'', ``Date'', and ``Filename''. The second column should
have the header ``Value'' and include the metadata values: ``EPA Air
Quality System (AQS)'', ``2018-2019'', and
``EPAair\_O3\_PM25\_NC1819\_Processed.csv''. The first column should be
aligned to the right and the second to the left.

\begin{longtable}[]{@{}ll@{}}
\caption{EPA Air Quality}\tabularnewline
\toprule\noalign{}
Item & Value \\
\midrule\noalign{}
\endfirsthead
\toprule\noalign{}
Item & Value \\
\midrule\noalign{}
\endhead
\bottomrule\noalign{}
\endlastfoot
Source & EPA Aur Quality System (AQS) \\
Date & 2018-2019 \\
Filename & EPAair\_O3\_PM25\_NC1819\_Processed.csv \\
\end{longtable}

\begin{center}\rule{0.5\linewidth}{0.5pt}\end{center}

\hypertarget{task-2---import-packages-and-data-suppressing-messages}{%
\subsection{Task 2 - Import packages and data, suppressing
messages}\label{task-2---import-packages-and-data-suppressing-messages}}

Set the following R code chunk so that it runs when knit, but no
messages, errors, or any output is shown. The code itself, however,
should be displayed.

\begin{Shaded}
\begin{Highlighting}[]
\CommentTok{\#Import libraries}
\FunctionTok{library}\NormalTok{(tidyverse);}\FunctionTok{library}\NormalTok{(lubridate);}\FunctionTok{library}\NormalTok{(here);}\FunctionTok{library}\NormalTok{(knitr);}\FunctionTok{library}\NormalTok{(tinytex)}

\CommentTok{\#Import EPA data (from the processed\_KEY folder) \& fix dates}
\NormalTok{epa\_data }\OtherTok{\textless{}{-}} \FunctionTok{read.csv}\NormalTok{(}
  \FunctionTok{here}\NormalTok{(}\StringTok{"Data"}\NormalTok{,}\StringTok{"Processed\_KEY"}\NormalTok{,}\StringTok{"EPAair\_O3\_PM25\_NC1819\_Processed.csv"}\NormalTok{),}
  \AttributeTok{stringsAsFactors =} \ConstantTok{TRUE}\NormalTok{) }\SpecialCharTok{\%\textgreater{}\%} 
  \FunctionTok{mutate}\NormalTok{(}\AttributeTok{Date =} \FunctionTok{ymd}\NormalTok{(Date))}
\end{Highlighting}
\end{Shaded}

\begin{center}\rule{0.5\linewidth}{0.5pt}\end{center}

\hypertarget{task-3-creating-tables}{%
\subsection{Task 3: Creating tables}\label{task-3-creating-tables}}

Set the following R code chunk to display two tables, using knitr's
\texttt{kable()} function, one listing the mean PM2.5 concentrations for
each county, and the other the same except for Ozone. The titles should
be ``Mean Particulates (2.5mm)'' and ``Mean Ozone'', respectively. And
the column names should be ``County'' and ``µg/m3'' for both tables.

Customize the chunk options such that the code is run but is not
displayed in the knitted document. The output, however, should be
displayed.

\begin{quote}
\textbf{TIPS:}

\begin{itemize}
\item
  Use \texttt{"\$\textbackslash{}\textbackslash{}mu\ g/m\^{}3\$"} as a
  column name to generate a nicely formatted string via markdown/MathJax
  notation
\item
  If your output table spans across two pages, try inserting a page
  break in the markdown just before your code chunk.
\end{itemize}
\end{quote}

\begin{longtable}[]{@{}lr@{}}
\caption{Mean Particulates (2.5mm)}\tabularnewline
\toprule\noalign{}
County & \(\mu g/m^3\) \\
\midrule\noalign{}
\endfirsthead
\toprule\noalign{}
County & \(\mu g/m^3\) \\
\midrule\noalign{}
\endhead
\bottomrule\noalign{}
\endlastfoot
Haywood & 13.98400 \\
New Hanover & 15.60681 \\
Avery & 18.27941 \\
Edgecombe & 26.06503 \\
Pitt & 27.37166 \\
Guilford & 29.14163 \\
Swain & 30.62780 \\
Johnston & 33.02695 \\
Durham & 33.53770 \\
Mecklenburg & 33.63038 \\
Forsyth & 35.09282 \\
Wake & 37.45423 \\
\end{longtable}

\begin{longtable}[]{@{}lr@{}}
\caption{Mean Ozone}\tabularnewline
\toprule\noalign{}
County & \(\mu g/m^3\) \\
\midrule\noalign{}
\endfirsthead
\toprule\noalign{}
County & \(\mu g/m^3\) \\
\midrule\noalign{}
\endhead
\bottomrule\noalign{}
\endlastfoot
Swain & 35.58367 \\
Avery & 38.39308 \\
Wake & 38.61345 \\
New Hanover & 39.11688 \\
Edgecombe & 39.22154 \\
Johnston & 40.33849 \\
Mecklenburg & 40.45746 \\
Durham & 40.69882 \\
Pitt & 41.64147 \\
Forsyth & 44.02352 \\
Haywood & 44.75049 \\
Guilford & 45.86681 \\
\end{longtable}

\begin{center}\rule{0.5\linewidth}{0.5pt}\end{center}

\hypertarget{task-3-plots}{%
\subsection{Task 3: Plots}\label{task-3-plots}}

Create two separate code chunks that create boxplots of the distribution
of Ozone levels by month using, one for only records collected in 2018
and one for records in 2019. Customize the chunk options such that the
final figures are displayed but not the code used to generate the
figures. In addition, align the plots on the left side of the page and
set the figure heights so both plots fit on the same page with minimal
space remaining. Lastly, add a \texttt{fig.cap} chunk option to add a
caption (title) to your plot that will display underneath the figure.

\begin{verbatim}
## Warning: Removed 1199 rows containing non-finite values (`stat_boxplot()`).
\end{verbatim}

\begin{figure}

\includegraphics{MaiaGriffith_A06_Lab_Crafting_Reports_files/figure-latex/Ozone 2018-1} \hfill{}

\caption{Ozone Levels by Month (2018)}\label{fig:Ozone 2018}
\end{figure}

\begin{verbatim}
## Warning: Removed 947 rows containing non-finite values (`stat_boxplot()`).
\end{verbatim}

\begin{figure}

\includegraphics{MaiaGriffith_A06_Lab_Crafting_Reports_files/figure-latex/Ozone 2019-1} \hfill{}

\caption{Ozone Levels by Month (2018)}\label{fig:Ozone 2019}
\end{figure}

\begin{center}\rule{0.5\linewidth}{0.5pt}\end{center}

\newpage

\hypertarget{task-4-knit-and-submit.}{%
\subsection{Task 4: Knit and submit.}\label{task-4-knit-and-submit.}}

Add a table of contents to your document and knit to a PDF. Submit your
PDF to Sakai, but also be sure to commit and push your Rmd file used to
create this knit document to GitHub. In the section below, add a link to
your GitHub repository.

\hypertarget{git-repository}{%
\subsection{Git Repository}\label{git-repository}}

\end{document}
